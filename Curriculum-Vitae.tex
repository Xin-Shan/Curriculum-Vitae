%% start of file `template.tex'.
%% Copyright 2006-2013 Xavier Danaux (xdanaux@gmail.com).
%
% This work may be distributed and/or modified under the
% conditions of the LaTeX Project Public License version 1.3c,
% available at http://www.latex-project.org/lppl/.


\documentclass[11pt,a4paper,sans]{moderncv}         % possible options include font size ('10pt', '11pt' and '12pt'), paper size ('a4paper', 'letterpaper', 'a5paper', 'legalpaper', 'executivepaper' and 'landscape') and font family ('sans' and 'roman')

% modern themes
\moderncvstyle{banking}                             % style options are 'casual' (default), 'classic', 'oldstyle' and 'banking'
\moderncvcolor{blue}                                % color options 'blue' (default), 'orange', 'green', 'red', 'purple', 'grey' and 'black'
%\renewcommand{\familydefault}{\sfdefault}          % to set the default font; use '\sfdefault' for the default sans serif font, '\rmdefault' for the default roman one, or any tex font name
%\nopagenumbers{}                                   % uncomment to suppress automatic page numbering for CVs longer than one page

% character encoding
\usepackage[utf8]{inputenc}                         % if you are not using xelatex ou lualatex, replace by the encoding you are using
%\usepackage{CJKutf8}                               % if you need to use CJK to typeset your resume in Chinese, Japanese or Korean

% adjust the page margins
\usepackage[scale=0.775]{geometry}
%\setlength{\hintscolumnwidth}{3cm}                 % if you want to change the width of the column with the dates
%\setlength{\makecvtitlenamewidth}{10cm}            % for the 'classic' style, if you want to force the width allocated to your name and avoid line breaks. be careful though, the length is normally calculated to avoid any overlap with your personal info; use this at your own typographical risks...

\usepackage{import}
\usepackage{{fontawesome}}

% personal information
\name{XIN}{SHAN}
\title{Curriculum Vitae}                            % optional, remove / comment the line if not wanted
\address{Engineering Building, 98 Brett Rd, Piscataway, NJ 08854}{}{}% optional, remove / comment the line if not wanted; the "postcode city" and and "country" arguments can be omitted or provided empty
\phone[mobile]{+1 (848) 391 1729}                   % optional, remove / comment the line if not wanted
%\phone[fixed]{01234 123456}                        % optional, remove / comment the line if not wanted
%\phone[fax]{+3~(456)~789~012}                      % optional, remove / comment the line if not wanted
\email{xin.shan@rutgers.edu}                          % optional, remove / comment the line if not wanted
\homepage{https://xin-shan.github.io/Curriculum-Vitae/}% optional, remove / comment the line if not wanted
%\extrainfo{\textbf{}}
%\photo[64pt][0.4pt]{picture}                       % optional, remove / comment the line if not wanted; '64pt' is the height the picture must be resized to, 0.4pt is the thickness of the frame around it (put it to 0pt for no frame) and 'picture' is the name of the picture file
%\quote{}                                           % optional, remove / comment the line if not wanted

\begin{document}
%\begin{CJK*}{UTF8}{gbsn}                           % to typeset your resume in Chinese using CJK

\makecvtitle

\section{Education Background}

\vspace{4pt}

\begin{itemize}

    \item{\cventry{9/2018--present}{Doctor of Philosophy}{Rutgers University}{New Brunswick, USA}{Mechanical and Aerospace Engineering}{\textit{Advisor: Dr. Onur Bilgen}}}  % arguments 3 to 6 can be left empty

    \vspace{4pt}

    \item{\cventry{9/2015--6/2018}{Master of Science}{Zhejiang University}{Hangzhou, China}{Mechatronic Engineering}{\textit{Exam-exempted recommended student \newline Advisor: Prof. Canjun Yang}}}  % arguments 3 to 6 can be left empty

    \vspace{4pt}

    \item{\cventry{9/2011--6/2015}{Bachelor of Engineering}{Central South University}{Changsha, China}{Mechanical Engineering}{\textit{Overall weighted average: 90/100 (Top 2\% student) \newline Advisor: Prof. Xiaoqian Li, Prof. Yongcheng Lin}}}

    %Ranked 8\textsuperscript{th} out of 402

\end{itemize}

\vspace{3pt}

\section{Publications}

\vspace{3pt}

\subsection{Journals}   

\begin{itemize}
        
    \item{\textbf{Shan, X.,} Yang, C., Wu, S., Chen, Y., Zhou, P., (2017). Integrated underwater optical guiding and communicating devices between an AUV and sea network nodes. \textit{Ocean Engineering.} (Under Review)}

    \vspace{3pt}

    \end{itemize}

    \begin{itemize}

    \item{Zhu, Y., Yang, C., Wu, S., Zhou, P., \& \textbf{Shan, X.} (2016). Steering performance of underwater glider in water column monitoring. \textit{Journal of Zhejiang University (Engineering Science)}, 50(9), 1637-1645.}

    \vspace{3pt}

\end{itemize}

\subsection{Conference Proceedings}   

    \begin{itemize}

    \item{\textbf{Shan, X.,}  Yang, C., Chen, Y., Xia, Q. (2017, November). A free-space underwater laser communication device with high pulse energy and small volume. In OCEANS'17 MTS/IEEE Anchorage. IEEE. (Presenter)}

    \vspace{3pt}

    \end{itemize}

    \begin{itemize}
        
    \item{Xia, Q., Chen, Y., Zang, Y., \textbf{Shan, X.,}  Yang, C., Zhang, Z. (2017, November). Ocean profiler power system driven by temperature difference energy. In OCEANS'17 MTS/IEEE Anchorage. IEEE.}

    \vspace{3pt}

\end{itemize}

\subsection{Patents} 

    \begin{itemize}

    \item{Yang, C., Hua, X., Wu, S., \textbf{Shan, X.,} Zhou, P., Zhi, H., Chen, Y. (2018). An integrated device for deep sea optical communication and track. \textit{CN109245821A.} (Pending)}

    \vspace{3pt}

    \item{Yang, C., Zhou, P., Wu, S., Zhu, Y., \textbf{Shan, X.,} \& Xu, X. (2016). Releasable bottom sitting device for underwater profiler. \textit{CN106197384A.}}

    \vspace{3pt}

    \item{Liu, G., \textbf{Shan, X.,} Feng, Z., Xu, N., Qin, Z., ... \& Lu, Z. (2015). An comprehensive processing system of automobile exhaust . \textit{CN204099004U.}}

    \vspace{3pt}

\end{itemize}

%\section{Standardized Test}

    %\vspace{2pt}
    
    %\begin{itemize}

    %\item{\cvdoubleitem{GRE General Test}{\hfill \textbf{327+3} \qquad \qquad \newline Verbal Reasoning: \hfill 159 \qquad \qquad \newline Quantitative Reasoning: \hfill 168 \qquad \qquad \newline Analytical Writing: \hfill 3 \qquad \qquad}{TOEFL IBT}{\hfill \textbf{97} \newline \qquad Reading: \hfill 28 \newline Listening: \hfill 23 \newline Speaking: \hfill 22 \newline Writing: \hfill 24}}

%\end{itemize}

\section{Thesis \& Projects}

    \vspace{3pt}

    \begin{itemize}

    \vspace{5pt}

    \item{\textbf{Master Thesis (2016-2018): }\textit{'Research on Laser Information and Energy Transmission Between Sea Observatory Network and AUV.'}

    \vspace{3pt}

    \small{This thesis aimed to integrate the communication and positioning functions into one portable device, which can guide underwater vehicles to the nodes of sea observatory network and allow the information and energy exchanging between them. I wrote an algorithm (FPGA Verilog), designed the hardware, and did the thermal and optical design for this system. A preliminary experiment of underwater laser charging was also included.}}

    \vspace{6pt}

    \item{\textbf{National High-Tech R\&D Program of China (No.2014AA09A513-1): }\textit{'Monitoring and Networking Technology of Autonomous Portable Profiler'}

    \vspace{3pt}

    \small{As a team member, my attribution was mainly on designing of main control board hardware (PCB), data logging program (STM32 Platform / C++Programing Language) and some auxiliary devices (deploy and recovery devices, etc.). I also worked with my team to assemble and debug the profiler. I attended the sea trial in July 2017 cooperated with 715 Institute of China Shipbuilding Industry Corporation.}}

    \vspace{6pt}

    \item{\textbf{The National Key R\&D Program of China: }\textit{'Development of Air-Tight Sampling Device for Deep Sea Water and Sediments'}

    \vspace{3pt}

    \small{The research on developing an extremely low power consumption optical communication system for the deep sea sampler was conducted in this project.}}

    \vspace{6pt}

    \item{\textbf{Bachelor Thesis (2014-2015): }\textit{'Design of Multi-Physical Field Aluminum Alloy Hemi-Continuous Casting Machine — Aluminum Liquid Level Control Systems'}

    \vspace{3pt}

    \small{I investigated the factory site and designed a hemi-continuous casting system. Moreover, I established mathematics and control model for the system and wrote program (PLC) of it, and designed the structure of the actuator.}}

    \vspace{6pt}

    \item{\textbf{Central South University Undergraduate Free Exploration Program (2013-2014): }\textit{'Study on Nonlinear Dynamic Response of Spur Gears with Clearance and Friction'}

    \vspace{3pt}

    \small{This was a university funded project, of which I am the PI of a 5-student group. We established the dynamic model of the gear and get the destruction mechanism and the allowable extreme working conditions.}}

    \vspace{6pt}

    \item{\textbf{National Contest of Energy Saving \& Emission Reduction (2012-2013): }\textit{'Comprehensive Purification System for Automobile Exhaust Based on Seebeck Effect and PM2.5 Electrostatic Removal Technology'}

    \vspace{3pt}

    \small{This was a national contest held by China's ministry of education. I was a key member of a 5 people group, who was in charge of the design of an electricity generation system. We won a national second prize in this contest.}}

    \end{itemize}

    \section{Technical Skills}

    \vspace{3pt}

    \begin{itemize}

    \item \textbf{Programming Language:} VHDL, Verilog, C, C++, Assembly Language (Intel 8051 Platform), LabVIEW, \LaTeX, Matlab, Maple, Mathematica, etc.

    \vspace{2pt}

    \item \textbf{Software:} COMSOL, ANSYS, Altium Designer, Xilinx ISE (FPGA Development Environment), TracePro (Optical Simulation Software), AutoCAD, SolidWorks, KeyShot, 3Ds Max, Catia, Pro/E, Visual Studio, Keil, Multisim, etc.

\end{itemize}

\section{Fellowships \& Awards}

\vspace{3pt}

\begin{itemize}

    \item{\cventry{09/2018}{Teaching Assistant of 'Dynamics' and 'Aircraft Flight Dynamics'}{Teaching Assistantship}{Rutgers University}{}{}}  

    \item{\cventry{12/2016}{Awarded to excellent graduate students}{Outstanding Student}{Zhejiang University}{}{}}  

    \item{\cventry{6/2015}{Awarded to excellent undergraduate students}{Outstanding Graduate}{Central South University}{}{\textit{}}}

    \item{\cventry{10/2013}{Awarded to top students}{National Scholarship}{Ministry of Education}{}{\textit{}}}

\end{itemize}

\end{document}
